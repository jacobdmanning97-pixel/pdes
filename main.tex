\documentclass{beamer}
%%%%%%%%%%%%%%%%%%%%%%%%%%%%%
% Do not change the next few lines. 
\usepackage{amsmath,amssymb,bigints,amsthm,mathrsfs,mathtools,biblatex,xcolor}

\addbibresource{bib.bib}

\newcommand{\singleslidebig}[1]{\begin{frame}
\begin{center}
\textcolor{red}{\Huge{#1}}
\end{center}
\end{frame}}

%%%%%%%%%%%%%%%%%%%%%%%%%%%%%

\usetheme{Warsaw}

\newtheorem{thrm}{Theorem}
\newtheorem{cor}{Corollary}
\newtheorem{lmma}{Lemma}
\newtheorem{define}{Definition}
\newtheorem{prop}{Proposition}
\newtheorem{nte}{Remark}
\newtheorem{pf}{Proof}
\newtheorem{ex}{Example}

\newcommand{\R}{\mathbb{R}}
\newcommand{\D}{\mathscr{D}}
\newcommand{\E}{\mathscr{E}}
\newcommand{\Sch}{\mathcal{S}}
\newcommand{\Lin}{\mathscr{L}}
\newcommand{\Om}{\Omega}
\newcommand{\W}[2]{W^{#1}(#2)}
\newcommand{\WO}[2]{W_0^{#1}(#2)}
\newcommand{\HM}[2]{H^{#1}(#2)}
\newcommand{\LP}[2]{L^{#1}(#2)}
\newcommand{\set}[2]{\{#1\:|\:#2\}}


\setbeamertemplate{enumerate items}[default]
\definecolor{myor}{RGB}{245,102,0}
\definecolor{mypur}{RGB}{82,45,128}

\setbeamercolor{structure}{fg=myor!90!mypur}
%\setbeamercolor*{block title example}{fg=white, bg= violet!90}
%\setbeamercolor*{block body example}{bg= violet!20}

\newtheorem*{model}{\footnotesize{Hodgkin-Huxley Model}}

%%%%%%%%%%%%%%%%%%%%%%%%%%%%%
% Now Start entering your information.
\title[$C_0$ Semigroups]{$C_0$ Semigroups}
\author[Manning]{\textbf{Jacob Manning}}
\institute[Clemson University]{Clemson University}
\date{April 16, 2025}

\begin{document}

\maketitle

\begin{frame}{Introduction}
    This presentation is based on three books, namely "Topics in functional analysis and applications" by Srinivasan Kesavan\cite{kesavan}, "Introductory functional analysis with applications" by Erwin Kreyszig\cite{kreyszig}, and "Semigroups of linear operators and applications to partial differential equations" by Amnon Pazy\cite{pazy}.
\end{frame}

\begin{frame}{Preliminary Material}
    \begin{define}
        \begin{itemize}
            \item A linear operator $A:D(A)\subseteq X\to Y$ is said to be \textbf{bounded} if there exists a $C>0$ such that
            \begin{align*}
                ||Au||_Y\leq C||u||_X,\quad \text{for every } u\in D(A)
            \end{align*}
            Otherwise it is said to be \textbf{unbounded}.
            \pause
            \item A linear operator $A:D(A)\subseteq X\to Y$ is said to be \textbf{densely defined} if $\overline{D(A)}=X$
            \pause
            \item A linear operator $A:D(A)\subseteq X\to Y$ is said to be \textbf{closed} if the \textbf{graph}
            \begin{align*}
                G(A)=\set{(u,Au)}{u\in D(A)}\subseteq X\times Y
            \end{align*}
            is closed as a subspace of $X\times Y$
        \end{itemize}
    \end{define}
\end{frame}

\begin{frame}{Preliminary Material}
    \begin{define}\
        Let $X$ be a Banach space with dual space $X'$. Denote $x'\in X'$ at $x\in X$ by $\langle x',x\rangle$ or $\langle x,x'\rangle$. Define the following set $F(x)\subseteq X'$ as
        \begin{align*}
            F(x)=\set{x'}{\langle x',x\rangle=||x||^2=||x'||^2}
        \end{align*}
        (This set is non-empty by the Hahn-Banach theorem.)
    \end{define}
    \pause
    \begin{define}[Dissipativity]\
        A linear operator $A$ is dissipative if for every $x\in D(A)$ there is a $x'\in F(x)$ such that $Re\langle Ax,x'\rangle\leq 0$
    \end{define}
    \pause
    \begin{define}[Maximal Dissipativity]\
        A linear operator $A$ is called maximally dissipative if it is dissipative and $R(I-A)=X$.
    \end{define}
\end{frame}

\begin{frame}{Preliminary Material}
        \begin{define}
        Let $X\neq \{0\}$ be a complex normed space and $A:D(A)\subseteq X\to X$ be a linear operator. With $A$ we associate the operator
        \begin{align*}
            A_\lambda =\lambda I-A
        \end{align*}
        where $\lambda$ is a complex number and $I$ is the identity operator on $D(A)$. If $A_\lambda$ has an inverse, we denote it by $R_\lambda (A)$ and call it the \textit{resolvent operator} of $A$ or, simply, the \textbf{resolvent} of $A$. If it is clear which operator we are discussing, we will write $R_\lambda$.
    \end{define}
\end{frame}

\begin{frame}{Preliminary Material}
    \begin{define}
        Let $X\neq \{0\}$ be a complex normed space and $A:D(A)\subseteq X\to X$ be a linear operator. A \textit{regular value} $\lambda$ of $A$ is a complex number such that
        \begin{itemize}
            \item $R_\lambda$ exists,\\
            \item $R_\lambda$ is bounded,\\
            \item $R_\lambda$ is densely defined.
        \end{itemize}
        The \textit{resolvent set} $\rho(A)$ of $A$ is the set of all regular values $\lambda$ of $A$. Its complement $\sigma (A)=\mathbb{C}\backslash\rho(A)$ in the complex plane $\mathbb{C}$ is called the \textit{spectrum} of $A$, and a $\lambda \in \sigma (A)$ is called a \textit{spectral value} of $A$.
    \end{define}
\end{frame}

\begin{frame}{Motivation}
    \begin{define}
        Recall the limit definition of the derivative
        \begin{align*}
            u_t=\lim_{h\downarrow 0}\frac{u(t+h)-u(t)}{h}
        \end{align*}     
    \end{define}
    \pause
    \begin{ex}
        Consider the following linear first order ODE
        \begin{align*}
            \begin{cases}
                u_t=-\alpha u \quad t>0,\: \alpha \in \R\\
                u(0)=u_0
            \end{cases}
        \end{align*}
    \end{ex}
\end{frame}

\begin{frame}{Motivation}
    \begin{lemma}
        $\log(1+x)=x\quad \text{as} \quad x\downarrow 0$\\
    \end{lemma}
    \pause
    \begin{pf}
        Let $y=\log(1+x)$.\\
        Note $y\to 0$ as $x\to 0$\\
        Thus,
        \begin{align*}
            1+x&=e^y\\
            &=1+y+o(y)\\
            &\to 1+y
        \end{align*}
    \end{pf}
    Now let's get back to the example
\end{frame}

\begin{frame}{Motivation}
    \begin{ex}[Continued]
        \begin{align*}
            -\alpha&=\frac{u_t}{u}\\
            \uncover<2->{&=\frac{1}{u(t)}\lim_{h\downarrow 0}\frac{u(t+h)-u(t)}{h}\\}
            \uncover<3->{&=\lim_{h\downarrow 0}\frac{1}{h}\log\bigg |1+\frac{u(t+h)-u(t)}{u(t)}\bigg|\\}
            \uncover<4->{&=\lim_{h\downarrow 0}\frac{1}{h}\log\bigg |\frac{u(t+h)}{u(t)}\bigg|\\}
            \uncover<5->{&=\lim_{h\downarrow 0}\frac{\log|u(t+h)|-\log|u(t)|}{h}\\}
            \uncover<6->{&=\frac{d}{dt}\log|u(t)|}
        \end{align*}
        \pause\pause\pause\pause\pause\pause
        Therefore $u(t)=u_0e^{-\alpha t}$
    \end{ex}
\end{frame}

\begin{frame}{$C_0$ Semigroup Definition}
    How can we generalize this idea?
    \begin{define}[$C_0$ Semigroups]
        Let $X$ be a Banach space and $\{S(t)\}_{t\geq 0}$ be a family of bounded linear operators on $X$. $\{S(t)\}_{t\geq 0}$ is said to be a $C_0$ semigroup if the following are true:
        \begin{itemize}
            \pause
            \item $S(0)=I$, the identity of $X$\\
            \pause
            \item $S(t+s)=S(t)S(s)$, for all $t,s\geq 0$\\
            \pause
            \item For every $u\in X$
                \begin{align*}
                    S(t)u\to u \quad \text{as } t\downarrow 0
                \end{align*}
        \end{itemize}
    \end{define}
\end{frame}

\begin{frame}{$C_0$ Semigroup Definition}
    \begin{define}
        Let $\{S(t)\}_{t\geq 0}$ be a $C_0$ semigroup on $X$. The \textbf{infinitesimal generator} of the semigroup is a linear operator $A$ given by
        \begin{align*}
            D(A)&=\bigg \{u\in X\:|\: \lim_{t\downarrow0} \frac{S(t)u-u}{t} \:\text{exists}\bigg \}\\
            Au&=\lim_{t\downarrow0} \frac{S(t)u-u}{t},\: u\in D(A)
        \end{align*}
    \end{define}
\end{frame}

\begin{frame}{Some $C_0$ Semigroup Properties}
    What are some properties of these semigroups?
    \pause
    \begin{thrm}
        Let $\{S(t)\}_{t\geq 0}$ be a $C_0$-semigroup on $X$. Then there exists $M\geq 1$ and $\omega$ such that
        \begin{align*}
            ||S(t)||\leq Me^{\omega t},\quad \text{for all } t\geq 0
        \end{align*}
    \end{thrm}
    \pause
    This is a direct result of the Uniform Boundedness Principle. Thus we have the following definition.
    \pause
    \begin{define}
        If $M=1$ and $\omega = 0$, so that $||S(t)||\leq 1$ for all $t\geq 0$, we say that $\{S(t)\}$ is a \textbf{contraction semigroup}.
    \end{define}
\end{frame}

\begin{frame}{Applications of $C_0$ Semigroups}
    How does this help us?
    \pause
    \begin{thrm}
        Let $\{S(t)\}_{t\geq 0}$ be a $C_0$ semigroup and let $A$ be its infinitesimal generator. Let $u\in D(A)$. Then
        \begin{align*}
            S(t)u\in C^1([0,\infty);X)\cap C([0,\infty);X)
        \end{align*}
        and
        \begin{align*}
            \frac{d}{dt}(S(t)u)=AS(t)u=S(t)Au
        \end{align*}
    \end{thrm}
\end{frame}

\begin{frame}{Applications of $C_0$ Semigroups}
    \begin{pf}
        Let $u\in D(A)$. Then
        \begin{align*}
            \uncover<2->{AS(t)u=\lim_{h\downarrow 0}\bigg (\frac{S(h)-I}{h}\bigg )S(t)u&=\lim_{h\downarrow 0}S(t)\bigg (\frac{S(h)-I}{h}\bigg )=S(t)Au\\}
            \uncover<3->{AS(t)u=S(t)&Au=D^+S(t)u.}
        \end{align*}
        \pause\pause\pause
        Next consider
        \begin{align*}
            \frac{S(t)u-S(t-h)u}{h}=S(t-h)\frac{S(h)u-u}{h}
        \end{align*}
        \pause
        \begin{align*}
            \frac{S(t)u-S(t-h)u}{h}-S(t)Au=S(t-h)\bigg (\frac{S(h)u-u}{h}-Au\bigg )\\+(S(t-h)-S(t))Au
        \end{align*}
    \end{pf}
\end{frame}

\begin{frame}
    \begin{pf}[Proof Continued]
        \begin{align*}
            \bigg |\bigg |S(t-h)\bigg (\frac{S(h)u-u}{h}-Au\bigg )\bigg |\bigg |&\leq Me^{\omega t}\bigg |\bigg |\frac{S(h)u-u}{h}-Au\bigg |\bigg |\\
            \uncover<2->{&\to 0 \quad \text{as}\:h\downarrow 0\\}
            \uncover<3->{||(S(t-h)-S(t))Au||&\to 0\quad \text{as}\:h\downarrow 0}
        \end{align*}
        \pause\pause\pause
        Thus,
        \begin{align*}
            D^-S(t)u=S(t)Au=D^+S(t)u
        \end{align*}
        \pause
        and
        \begin{align*}
            \frac{d}{dt}(S(t)u)=AS(t)u=S(t)Au
        \end{align*}
        \pause
        Similarly, by the boundedness of $S(t)$, the map $t\mapsto S(t)Au$ is continuous so $S(t)u\in C^1([0,\infty);X)$.
    \end{pf}
\end{frame}

\begin{frame}{Applications of $C_0$ Semigroups}
    What does that tell us?
    \pause
    \begin{nte}
        If $A$ is the infinitesimal generator of a $C_0$ semigroup $\{S(t)\}$ then we know by the above theorem that
        \begin{align*}
            u(t)=S(t)u_0
        \end{align*}
        defines the unique solution of the initial value problem
        \begin{align*}
            \begin{cases}
            \frac{du(t)}{dt}&=Au(t),\:t\geq 0\\
            u(0)&=u_0
            \end{cases}
        \end{align*}
    \end{nte}
\end{frame}

\begin{frame}{Hille-Yosida Theorem}
    When can we guarantee that an operator is an infinitesimal generator of a contraction semigroup?
    \pause
    \begin{thrm}
        A linear unbounded operator $A$ on a Banach space $X$ is the infinitesimal generator of a contraction semigroup if and only if
        \begin{itemize}
            \pause
            \item $A$ is closed\\
            \pause
            \item $A$ is densely defined\\
            \pause
            \item For every $\lambda>0,\: R_\lambda(A)$ is a bounded linear operator and
            \begin{align*}
                ||R_\lambda(A)||\leq \frac{1}{\lambda}
            \end{align*}
        \end{itemize}
    \end{thrm}
    \pause
    This is a more general result and will not be shown.
\end{frame}

\begin{frame}{Lumer-Phillips}
    In Hilbert spaces it might be easier to use the following theorem.
    \pause
    \begin{thrm}
        \begin{itemize}
            \item If $A$ is dissipative and there is a $\lambda _0 >0$ such that $R(A_{\lambda _0})=X$, then $A$ is the infinitesimal generator of a $C_0$ semigroup of contractions on $X$.
            \item If $A$ is the infinitesimal generator of a $C_0$ semigroup of contractions on $X$ then $R(A_{\lambda})=X$ for all $\lambda >0$ and A is dissipative.
        \end{itemize}
    \end{thrm}
\end{frame}

\begin{frame}{Lumer-Phillips}
    \begin{pf}
        Let $\lambda >0$, the dissipativeness of $A$ implies that
        \begin{align*}
            ||A_\lambda x||=||\lambda x-Ax||\geq \lambda ||x||\quad \text{for every }\lambda >0 \text{ and } x\in D(A).
        \end{align*}
        Since $R(A_{\lambda _0})=X$, it follows when $\lambda=\lambda_0$ that $R_{\lambda_0}$ is a bounded linear operator and thus closed. This implies $A$ is closed. If $R(A_{\lambda})=X$ for every $\lambda>0$ then $\rho(A)\supseteq (0,\infty)$ and $||R_\lambda||\leq \lambda^{-1}$. It follows by the Hille-Yosida theorem that $A$ is the infinitesimal generator of a $C_0$ semigroup of contractions on $X$.\\
        \pause
        Consider the set
        \begin{align*}
            \Lambda=\set{\lambda}{0<\lambda<\infty,\:R(A_{\lambda})=X}.
        \end{align*} 
        Let $\lambda \in \Lambda$. By the previous inequality, $\lambda\in\rho(A)$. Since $\rho(A)$ is open, the intersection of $B_r(\lambda)\cap\R\subseteq \Lambda$ and thus $\Lambda$ is open.\\
    \end{pf}
\end{frame}

\begin{frame}
    \begin{pf}[Proof Continued]
        On the other hand, let $\{\lambda_n\}\subseteq\Lambda$ and $\lambda_n\to\lambda>0$. For every $y\in X$ there exists an $x_n\in D(A)$ such that
        \begin{align*}
            A_\lambda x_n=\lambda_n x_n -Ax_n=y
        \end{align*}
        \pause
        From the inequality it follows that $||x_n||\leq \lambda_n^{-1}||y||\leq C$ for some $C>0$. Now,
        \begin{align*}
            \lambda_m||x_n-x_m||&\leq ||\lambda_m(x_n-x_m)-A(x_n-x_m)||\\
            &=|\lambda_n-\lambda_m|\:||x_n||\\
            &\leq C|\lambda_n-\lambda_m|\to 0
        \end{align*}
        We see $\{x_n\}$ is Cauchy. Let $x_n\to x$. It follows $Ax_n\to \lambda x-y$. Since $A$ is closed, $x\in D(A)$ and $A_\lambda x=y$. Thus $R(A_{\lambda})=X$ and $\lambda\in\Lambda$ which implies that $\Lambda$ is closed. \pause By assumption, $\Lambda\neq \emptyset$, therefore $\Lambda=(0,\infty)$\\
    \end{pf}
\end{frame}

\begin{frame}
    \begin{pf}[Proof Continued]
        If $A$ is the infinitesimal generator of a $C_0$ semigroup of contractions, $S(t)$, on $X$, then by the Hille-Yosida theorem $\rho(A)\supseteq (0,\infty)$ and therefore $R(A_{\lambda})=X$ for all $\lambda>0$.\pause \:Furthermore if $x\in D(A),\: x'\in F(x)$ then
        \begin{align*}
            |\langle S(t)x,x'\rangle|\leq ||S(t)x||\:||x'||\leq ||x||^2
        \end{align*}
        \pause
        and therefore,
        \begin{align*}
            Re\langle S(t)x-x,x'\rangle=Re\langle S(t)x,x'\rangle-||x||^2\leq 0.
        \end{align*}
        By dividing the previous line by $t>0$ and letting $t\downarrow 0$ yields
        \begin{align*}
            Re\langle Ax,x'\rangle\leq 0.
        \end{align*}
    \end{pf}
\end{frame}
    
\begin{frame}{Lumer-Phillips}
    By using our definition of maximally dissipative operators we can rewrite the Lumer-Phillips theorem as follows
    \pause
    \begin{thrm}
        A densely defined operator $A$ is the infinitesimal generator of a $C_0$ semigroup of contractions if and only if it is maximal dissipative.
    \end{thrm}
\end{frame}

\begin{frame}{Bibliography}
    \printbibliography[heading=none]
\end{frame}

\end{document}
